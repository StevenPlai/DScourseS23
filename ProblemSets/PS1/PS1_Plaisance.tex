\documentclass{article}
\usepackage[utf8]{inputenc}

\title{PS1}
\author{Steven Plaisance}
\date{January 2023}

\begin{document}

\maketitle

\section{Body}

\text {My personal interest in data science stems from a curious nature that I had long before I knew the first thing about data science, or statistics for that matter. For as long as I can remember, I've been doing mental calculations in an effort to make everyday processes more efficient. This can take many forms, such as analyzing the correct intersection to make a left turn when driving across town. I realized from an early age that our society necessitates constant repetition, to the degree that even the most marginal efficiencies can add up to entire days saved each year. To the return to the traffic example: many people take the exact same route to work every day. Imagine two people that live on the same block and work in the same building. Due to a slight difference in their routes to work, Person A arrives 1 minute sooner than Person B on average. Not a big deal right? But over the course of one year, assuming 260 working days per year, Person A will spend 8.7 less hours driving to and from work than Person B. That's a big deal. Ultimately, this is where my interest in data science stems from. I'm fascinated with the process of using evidence and statistics to maximize time, money, or wins. I had interest in this particular course after taking and thoroughly enjoying a previous course with Professor Ransom. In terms of project ideas, I'm sure I will gravitate to something revolving around American Football, though I haven't narrowed much further than that yet. My primary goal for this class is to have a good experience and to acquire foundational knowledge of several productivity software. My goal after graduation is is to continue doing whatever makes me happy. For now, I quite enjoy my current job in Sport Science for Oklahoma Football, but I'm sure there are plenty more things out there for me to experience when the time comes.}

\section{Equation}

\[ a^2 + b^2 = c^2 \]

\end{document}
